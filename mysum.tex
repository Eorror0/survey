\documentclass[journal,twoside,web]{ieeecolor}
\usepackage{tmi}
\usepackage{cite}
\usepackage{amsmath,amssymb,amsfonts}
\usepackage{algorithmic}
\usepackage{graphicx}
\usepackage{textcomp}
\def\BibTeX{{\rm B\kern-.05em{\sc i\kern-.025em b}\kern-.08em
    T\kern-.1667em\lower.7ex\hbox{E}\kern-.125emX}}
\markboth{\journalname, VOL. XX, NO. XX, XXXX 2020}
{Author \MakeLowercase{\textit{et al.}}: Preparation of Papers for IEEE TRANSACTIONS ON MEDICAL IMAGING}
\begin{document}
\title{Preparation of Papers for IEEE TRANSACTIONS ON MEDICAL IMAGING}
\author{First A. Author, \IEEEmembership{Fellow, IEEE}, Second B. Author,and Third C. Author, Jr., \IEEEmembership{Member, IEEE}
\thanks{This paragraph of the first footnote will contain the date on which
you submitted your paper for review. It will also contain support information,
including sponsor and financial support acknowledgment. For example, 
``This work was supported in part by the U.S. Department of Commerce under Grant BS123456.'' }
\thanks{The next few paragraphs should contain the authors' current affiliations,
including current address and e-mail. For example, F. A. Author is with the
National Institute of Standards and Technology, Boulder, CO 80305 USA (e-mail:author@boulder.nist.gov). }
\thanks{S. B. Author, Jr., was with Rice University, Houston, TX 77005 USA.
He is now with the Department of Physics, Colorado State University,
Fort Collins, CO 80523 USA (e-mail: author@lamar.colostate.edu).}
\thanks{T. C. Author is with the Electrical Engineering Department,
University of Colorado, Boulder, CO 80309 USA, on leave from the National
Research Institute for Metals, Tsukuba, Japan (e-mail: author@nrim.go.jp).}}

\maketitle

\begin{abstract}

\end{abstract}

\begin{IEEEkeywords}


Enter about five key words or phrases in alphabetical order, separated by commas.
Cancer survival analysis,Cancer survival prediction,deep learning,whole slide image
\end{IEEEkeywords}

\section{Introduction}
\label{sec:introduction}
% motivation 添加 


Survival prediction is a direction of statistics for analyzing the duration time that is expected until the events of interest occur,such as the death of the life form of biology.
% 癌症是当前人类面临的最大的健康危机之一,包括各种类型的癌症,乳腺癌,肠胃癌,肺癌等等。临床领域为了减少癌症的危害,可以通过使用wsi类型的切片,由anatomic pathologists使用haematoxylin and eosin (H&E)方法染色的一种组织切片,进行生存预测。生存预测的使用能使得临床医生能够更好地了解患者病情,给出更好的治疗方案,因此能够提高患者的存活率和存活年限。
With the rising importance of precision medicine, the ability to predict patient survival based on individualized characteristics becomes paramount.  WSI, as a rich source of information, offers an unprecedented level of granularity by capturing the heterogeneity within tissue samples. By enabling the high-resolution digitization of entire tissue slides, WSI has not only enhanced the efficiency and accuracy of pathological assessments but has also opened new avenues for predictive modeling and survival analysis.

This review aims to provide a comprehensive overview of the application of WSI in the context of survival prediction. We delve into the intersection of digital pathology and computational methods, exploring how WSI data can be harnessed to predict patient outcomes, especially in the field of oncology.

The introduction of WSI has given researchers and clinicians access to vast repositories of image data, offering new opportunities for feature extraction, machine learning, and deep learning techniques to model and predict patient survival. The ability to analyze tissue morphology, cellular structures, and patterns within whole slide images has the potential to uncover valuable insights into disease progression and prognosis.

In this review, we will examine the key methodologies, challenges, and recent advancements in survival prediction using WSI data. We will also discuss the implications of such predictions on personalized medicine, treatment strategies, and patient care.Additionally, we will explore various aspects of survival prediction using WSI, including the development of novel image features, the utilization of deep learning architectures for image analysis, and the challenges associated with large-scale data management.  We will also consider the ethical and regulatory considerations in implementing such predictive models in clinical practice.

As the fields of digital pathology and computational biology continue to evolve, the integration of WSI into survival analysis represents a promising frontier for improving patient outcomes and advancing our understanding of disease dynamics. This review will serve as a guide for researchers and practitioners interested in leveraging WSI for survival prediction in diverse clinical and research settings.

\section{Background}
% 癌症风险预测是指对潜在的病人的全切片WSI进行分析,也可以结合病人的其他临床数据,例如年龄、性别等,得出病人直至死亡的存活时间。
% 在生存预测问题中,所使用的数据集中,所使用的censorship和survival time基本都是右删失。
In recent years, several computer algorithms for hematoxylin and eosin (H\&E) stained pathology image analysis have been developed to aid pathologists in objective clinical diagnosis and prognosis.
The Cancer Genome Atlas (TCGA) are the most common datasets when discussing the topics of survival prediction.

the observation of one patient is either a survival time $(O_i)$ or a censored time $(C_i).$ If and only if $t_i=\min(O_i,C_i)$ can be observed during the study, the dataset is right- censored [17]. An instance in the survival data is usually represented as $(x_i,t_i,\delta_i)$ where $x_i$ is the feature vector, $t_i$ is the observed time, $\delta_i$ is the indicator which is 1 for a un- censored instance (death occurs during the study) and 0 for a censored instance.

The survival function $S(t|x)=Pr(O\geq t|x)$ is used to identify the probability of being still alive at time $t$ where $x=(x_1,...x_p)^T$ is the covariates of dimension $p$, The haz- ard function is defined as

In recent years, the integration of Whole Slide Imaging (WSI) into the realm of survival prediction has emerged as a promising frontier in the fields of digital pathology, oncology, and computational biology. WSI, a revolutionary technology, has transformed the way we analyze tissue samples, providing a comprehensive digital representation of entire pathology slides.

Traditionally, pathological assessments have been reliant on manual inspection and subjective interpretation of glass slides under a microscope. While this approach has served as the gold standard for disease diagnosis and prognosis, it is inherently limited by issues of inter-observer variability, labor intensiveness, and the inability to harness the full potential of the vast data embedded in tissue structures.

The advent of WSI has paved the way for a paradigm shift in tissue analysis. Through high-resolution digitization, WSI generates massive datasets of tissue samples, allowing for a more detailed examination of cellular structures, tissue morphology, and spatial relationships. These digital representations of tissue hold a wealth of information that extends far beyond what can be appreciated by the human eye.

One of the most compelling applications of WSI lies in the realm of survival prediction, particularly in the context of cancer. Cancer remains a leading cause of mortality worldwide, and the ability to predict patient outcomes accurately is paramount for optimizing treatment strategies and improving patient care. With the integration of WSI, researchers and clinicians can extract rich image features and employ advanced machine learning and deep learning techniques to model and predict patient survival.

Survival prediction using WSI offers the potential to uncover subtle patterns, biomarkers, and prognostic factors that may have gone unnoticed in traditional pathology assessments. The granularity and comprehensiveness of WSI data open doors to new avenues of research and personalized medicine, enabling clinicians to tailor treatments to individual patient profiles.

In this review, we delve into the intricate interplay between digital pathology and computational methods, exploring how WSI is being harnessed to predict patient survival. We will examine the methodologies, challenges, ethical considerations, and the implications of these predictions on clinical practice.

As the field of WSI continues to evolve, the fusion of digital pathology and survival prediction holds great promise for advancing our understanding of disease dynamics and ultimately enhancing patient outcomes.

As we delve deeper into this evolving landscape, it is crucial to appreciate the monumental shift brought about by WSI. Its capacity to create digital archives of pathology slides has not only expedited the diagnostic process but has also catapulted computational pathology to the forefront. This dynamic shift, driven by advancements in digital imaging and artificial intelligence, has redefined the scope of pathology by offering a more profound understanding of disease, bolstering diagnostic accuracy, and heralding a new era of prognostic modeling.

WSI's role in survival prediction is particularly significant within the domain of cancer research. It has given rise to the emergence of predictive models that harness the extensive image data contained within tissue samples. By analyzing and extracting pertinent information from these digitized slides, researchers can identify morphological and structural biomarkers, which, when combined with clinical data, offer valuable insights into patient outcomes.

Moreover, the integration of WSI data provides the foundation for the development of prognostic models capable of guiding treatment decisions and improving the overall quality of patient care. The implications extend beyond individualized medicine to population-level studies, ultimately influencing healthcare policy, resource allocation, and public health strategies.

While the potential is immense, it is not without challenges. The management and analysis of vast WSI datasets demand advanced computational infrastructure, robust machine learning algorithms, and a harmonious partnership between pathologists and data scientists. Ethical considerations surrounding patient data privacy, model interpretability, and regulatory compliance also merit attention.

This review embarks on a journey to explore the multifaceted landscape of survival prediction using WSI. It endeavors to survey the methodologies, showcase the milestones achieved, and address the formidable challenges that lie ahead. In doing so, it aspires to serve as a guiding compass for researchers, pathologists, clinicians, and healthcare stakeholders invested in leveraging the symbiotic relationship between digital pathology and survival prediction.

As we advance in this ever-evolving field, the fusion of digital pathology with survival prediction promises to unravel the intricacies of disease dynamics, enhance patient care, and pave the way for a new era of data-driven medicine.

\section{Methodology}
\subsection{Region of Interest}
There exist many methods developed for predicting survival through the information provided by the whole slide images(WSIs).
Rather than utilizing the overall patches from the gigapixel pathology images, the traditional models usually pre-select a subset of critical patches from the region of interest (ROI) as the input data.
Apart from the features extracted from ROIs with the deep neural networks(DNNs), some morphological features of the image patches can be extracted and accessed by the image analysis software named CellProfiler\cite{lamprecht2007cellprofiler}, commonly used for cell phenotypes measurements at that time.
It's the features including cell shape, size, the distribution of pixel intensity in the cells and nuclei and texture of cells and nuclei that the quantitative analysis tool can extract.
% DeepConvSurv
% To evaluate whether
Taking whole slide images with various size as inputs, in order to do end-to-end survival prediction, DeepConvSurv proposed by \cite{zhu2016deep} randomly chooses the patches among the ROIs annotated by the professional pathologists.
And the experiments showed it could extract more abstract information different from the hand-crafted features generated by the state-of-the-art analysis tool CellProfiler mentioned above.
% convnet [6], [8], [9], [10], [11]
Except this, \cite{zhu2017wsisa}, \cite{di2020ranking}, \cite{yao2020whole}, \cite{abbet2020divide}, \cite{yao2019deep} have shown that random sampling of patches within the tissues in WSIs still makes sense in stratifying phenotypic information which can be improved.
% Predicting cancer outcomes
% in this paper;results;
With the help of online tool, Mobadersany et al.\cite{mobadersany2018predicting} manually selected the ROIs without tissue-processing artifacts containing overstaining or understaining areas from alternates, which have viable tumor features. 
Faced the difficulty of intratumoral heterogeneity and few availability of labeled data, to obtain better and more robust effects, the model uses the data augmentation techniques and sorted median risks to get prediction results.
And the model eventually gets more accurate outcomes. 
Moreover, the paper provides some publicly accessible datasets with ROIs.
% Pathomic Fusion;microenvironment 
With the assistance of the assembled diagnostic slides offered by \cite{mobadersany2018predicting} with ROIs, characteristics of tumor microenvironment could be got in the built graphs in \cite{chen2020pathomic}.
Because semantic segmentation is executed in ROIs to recognize and localize relevant cells acting as set of nodes in the spatial graph for abstract graph representations.
% deepcorrsurv
Not only dose the model in \cite{yao2017deep} use the public cancer survival dataset TCGA, but it also adopts a core sample set from UT MD Anderson Cancer Center during the holistic procedure.
And it takes advantage of the annotations of ROIs to locate the possible tumor regions in pathological images for subsequent steps.
% Comprehensive analysis
% In this study;In the NLST dataset;
% esat中提到分类
Some methods adopt sampling strategy to generate candidate patches not limited to ROIs.
Since revealing ROIs requires specialized prior knowledge and expensive labor costs, Wang et al. proposed an automatic model aimed at finding ROIs.
The proposed model in \cite{wang2018comprehensive} can identify tumor regions as ROIs in hematoxylin and eosin (H\&E) stained pathology images using predicted likelihood of image patches, each patch tagged as the highest probability category. 
In this way, some tumor area-related features can be extracted as the descriptors of above ROIs including area, perimeter, convex area, filled area, major axis length, minor axis length and so on.
For the purpose of training the prognostic model indicating that the risk group defined by tumor shape features is an independent prognostic factor, the features are used in the training process.
% capsurv
The model in \cite{mackenzie2022neural} has a automated pipeline excluding the background white space among the identified tissue, then overlooking the sparse cellularity regions and randomly sampling the potential patches from the foreground area.

In a short, as the methods previously implemented, ROIs routinely ask artificial marking and rigorous reviews to produce passable survival prediction.
Additionally speaking, the ROI-based methods discussed above require pathologists to hand-annotate ROIs, a tedious task.
\subsection{Feature Extraction}
Previous methods often extract image features from patches of whole slide images(WSIs) using the pre-trained model based on the numerous natural images datasets ImageNet theoretically being able to withdraw the low-dimensional features such as edge and texture.However,they have ignored the enormous difference between the WSIs and the natural images.
Recently, some new methods have been proposed to suitably get features from the WSIs to overcome the significant shortcomings.
Without the knowledge of each patch-level labels, the self-supervised learning methods can autonomously impart the outstanding feature extraction ability to the model.

% Self supervised contrastive
% abstract
The method named SimCLR in \cite{ciga2022self}, one of the self-supervised learning(SSL) methods, using the contrastive learning has excellent feature extraction capacity even comparable to the supervised learning model thanks to the collected 57 digital histopathology datasets with none labels.
% Integration of Patch;Train Feature Extraction Model;Recent studies have shown;
SeTranSurv, the proposed model in \cite{huang2021integration} applies SimCLR to train the initial feature extraction model ResNet18\cite{he2016deep} to get the specialized model for downstream task survival prediction.
During the training the model applies the contrastive loss\cite{he2020momentum} to enhance feature extraction ability.
Firstly augmentation module in SeTranSurv using two methods to transform a random image example to a position pair, then the model uses the fine-tuned feature extractor raised above to get the features. 
Considering other images as negative examples ensures that the views of different slide images are far apart in the high-dimensional space and the views of the same image are closer in the process of training.
Additionally, position encodings capturing spatial information and self-attention modules learning correlation between patches are put into the training of the model above to obtain slide-level features and therefore the patient-level features.
% Fine-Tuning and training
% higt,hgcn直接用;hyper adac,ds-mil用simclr作为策略
Aside from SimCLR as a strategy used in \cite{benkirane2022hyper} and \cite{li2021dual}, KimiaNet from \cite{riasatian2021fine} has also been one of the most welcome pre-trained models used for survival analysis exploiting a variable, multi-organ open image repository lick TCGA, which has been employed directly to extract the feature embeddings of image patches in \cite{guo2023higt} \cite{li2021dual}.
% Cancer Survival Prediction即convnet
% Feature Extraction Via
Without using convnet-based methods, the model in \cite{fan2022cancer} also considers the self-supervised learning (SSL) methods mainly for making full use of plentiful color information designed for WSI patches or pixels without hand-actuated labels, totally colorization and cross-channel as the pretext tasks.
The colorization model is trained to predict corresponding color channels based on the lightness channel and the letter, on the other hand, is trained to get lightness channel using given color channels data, after which the visualized results indicating highlighted overall structure of nuclei and tissue.

Unlike prior studies that pick and choose specific regions of interest (ROIs) from whole-slide images (WSIs) for survival analysis, new method can utilize the complete tissue and tumor micro-environment without the need for detailed annotations within WSIs. Additionally, it can seamlessly incorporate multiple diagnostic slides of various dimensions from a single patient sample for both training and inference in a unified framework.

\subsection{Multimodal}
One limitation for some existing survival models is that they initially focus on one modality and cannot sufficiently handle multi-modalities data. 
Actually, multi-modalities information could provide complementary and auxiliary information for tumor diagnosis.
For instance, molecular data and whole slide images share relevant representations to describe the same event in tumor growth and symptoms which are very critical for tumor diagnosis.
Therefore,it is essential and necessary to combine and integrate multi-modal data such as pathological images, genotypic information and clinical data for explaining and understanding cancerous heterogeneity and complex symptoms for customized treatments and healings, consequently boosting the survival predictions.
Although hematoxylin and eosin (H\&E)-stained slides are enough to build a comprehensive diagnosis, other modal data can provide a deeper description of the tumor. For example, genomic profiles being comprising of ten thousand dimensional sequences can provide a molecular characterization of the tumor.
Additionally,during the past several decades, multiple clinicians have made clinical cancer survival prediction on the basis of clinical variates and experience, therefore the clinical data is also an important source modality for multi-modal survival prediction.
However, multi-modal survival prediction faces an important challenge due to the huge data heterogeneity gap between WSIs and other modalities, and many proposed approaches use simple multi-modal fusion mechanisms for feature incorporation, which give up mining important multi-modal relationships.
Unlike needle-in-a-haystack challenges, survival outcome prediction involves the modeling of diverse visual elements within the tumor micro-environment. 
These elements are not distinguishable by traditional Multiple Instance Learning (MIL) methods. 
For instance, they include the co-occurrence of tumor cells and lymphocyte infiltrates, which is linked to a positive prognosis. 
This necessitates the modeling of interactions over medium to long distances among instances within Whole Slide Images (WSI).
While it is commonly tackled as a weakly-supervised problem utilizing solely gigapixel Whole Slide Images (WSIs), the conventional perspective on survival outcome prediction views it as a multi-modal learning task.
Due to the variant costs and trauma degree of multi-omics examination, the scenarios of missing modalities are very common.  

% LUNG CANCER SURVIVAL PREDICTION
The model in \cite{zhu2016lung} gets the features from the pathological images extracted using Cellprofiler\cite{lamprecht2007cellprofiler} from the image tiles in ROIs,namely geometry, texture, holistic features.In addition,the model uses the preprocessed genetic data and one feature selection operation SPACE\cite{peng2009partial} to select representative features to integrate data for the principal component regression model for survival prediction.
The experiments focusing on ADC lung cancer revealed that the results could be better than only using data of genes or images, which demonstrated that the genetic data could actually enhance the prediction performance of the survival analysis model to some extent.

% DeepCorrSurv
Differently, the experiments in \cite{yao2017deep} were conducted on other two cancer types: glioblastoma multiforme (GBM) and lung squamous cell carcinoma (LUSC) using pathological images and molecular data including protein data, Copy number variation(CNV) data and so on. 
But similarly, to fuse the data from different modality for better results, the model firstly learn deep representations from two kinds of data using separate Convolutional Neural Networks (CNNs).Next, the representations passing through the sub-network in the model are connected to get the new representation, which serves as the input of the correlational layer.
Certainly, the deep correlation layer is used to decrease the discrepancy by maximizing the correlation.
After the common layer, the output acts as the input of the survival prediction layer using the negative log partial likelihood as survival loss function.
Compared with the models handling the linear condition, the new model DeepCorrSurv can learn complex correlation using deep neural networks by using the unsupervised method to learn the interactions and the survival loss network to fine-tune the model, eventually getting better results in the comparison experiments about lung and brain cancer.
The results showed that the common representation after maximizing step could bring better performance to the survival prediction measured by the metric named concordance index values or c-index values.

% pnas
In the paper\cite{mobadersany2018predicting}, the model named GSCNN aims for fusing genomic data and second modal data from The Cancer Genome Atlas (TCGA) Lower-Grade Glioma (LGG) and Glioblastoma (GBM) projects. 
If the second modal data are added to the network during the whole training process, the median c index will improve more than simply integrating the second type data into the fully connected layers. Then the model proves that the Molecular subtype is significant in the multi-variable regression model.

% GPMKL
Again, for genomic information, Sun et al. in \cite{sun2018integrating} considered building distinct models to make survival predictions according to the five major molecular subtypes of the breast cancer.
The genomic data chosen in the model consists of gene expression, copy number alteration (CNA), gene methylation and protein expression.
The main idea in the paper is how to merge different types of data as one type, and in this way using multiple kernel learning (MKL) is a choice.
To integrate the genotypic information and image data, GPMKL model was proposed using 5 independent Gaussian kernels and having a integrating step.
The baseline algorithms used that time for comparison includes LASSSO-Cox\cite{tibshirani1997lasso}, elastic-net penalized Cox (EN-Cox)\cite{yang2013cocktail}, Parametric censored regression models (PCRM)\cite{kalbfleisch2011statistical}, Random survival forests (RSF)\cite{ishwaran2008random}, Boosting concordance index (BoostCI)\cite{mayr2014boosting}, Supervised principal components regression (superPC)\cite{bair2006prediction}.
Additionally, two independent models named GMKL and PMKL was constructed only adopting genomic data or pathological images data and four single dimensional models using four types of genomic data were also built. 
Subsequently, the results showed that the gene expression and protein information play relatively more important role than others and CNA makes a little contribution to holistic prediction accuracy.

% MultimodalPrognosis
In the model proposed in \cite{cheerla2019deep} named MultimodalPrognosis, clinical, genomic and WSI images data are processed by FC layers, deep highway networks\cite{srivastava2015highway}, the SqueezeNet architecture\cite{iandola2016squeezenet} separately.
The microRNA data is also passed through deep highway networks discussed. 
One notable problem about the microRNA and clinical data is the missing data.
However, the function of the highway networks dose not stand out without comparison with other methods in the paper.
Then, the similarity loss is used to train the model to recognize the patient-distinguishing patterns and correspond the data from different modality to generate associated representations. 
Thus, the final loss function is composed of cox loss and similarity loss in the unsupervised model.
The multi-modal dropout was also invented that is dropping whole feature vectors of each modality and accordingly increase the weights of other modalities to build robust representations.
The solution above was then validated in the experiments including visualizing the encodings of the pancancer patient cohort and calculating the C-index values and the results also demonstrated the essence of the adopted modalities.

% OSCCA
One more variation usually ignored is the ordinal relationship among the survival time of different patients, which is assumed independent among patients.
The model proposed in \cite{shao2018ordinal} named OSCCA intends to take advantage of the information.
For extracting features, the model uses the methods in \cite{phoulady2016nucleus} to generate segmented nucleus and get the specific features.
As to gene expression data, the co-expression network analysis algorithms are utilized to derive eigen-gene features.
Considering the correlation between imaging and eigen-gene features, sparse canonical correlation analysis(i.e. SCCA) is used to choose features.
Among the datasets, most patients are censored, denoting that their actual living time is longer than recorded data, while the uncensored patients have the real survival time.
To make full use of the censored state, an ordinal sparse canonical correlation analysis (OSCCA) method is proposed.
In the newly proposed method, the equation estimates the uncensored information, while linear inequalities restrains the ordered relationship between censored and uncensored data. 

% OMMFS
After the model above, Shao et al. developed a new model named OMMFS in \cite{shao2019integrative} using CNV data and level-3 DNA Methylation (DME) data additionally.
The experiments showed that the log-rank test\cite{cheng2017integrative} had better stratification performance than univariate Cox regression method.
Furthermore, the model implements the second feature selection function based on the Generalized Sparse Canonical Correlation Analysis (GSCCA) framework\cite{witten2009extensions} to get the inherent relationship among different modalities.
Likewise the modality in OSCCA, the model under discussing also notices the survival information of patients.
The validation experiments demonstrated that the new model even had superior stratification of early-stage KIRC patients.
To make comparison, the SGSCCA model was proposed with the same objective function without ordered survival information.
However, the feature pre-selection strategy is based on the median value, which is too arbitrary to consider the accurate relevance between features and cancer type.
Also, the image features relies on the regions of interest annotated by pathologists.

% PAGE-Net;gpt
This model in \cite{hao2019page} has a two-stage feature extraction process used in a deep learning model for pathology-specific layers. 
In the first stage, a pre-trained convolution neural network (CNN) is used to identify survival-discriminative features in patches of pathological images. 
These features are obtained through dilated convolution layers and max-pooling. 
In the second stage, global survival-discriminative features for a whole slide image (WSI) are generated by aggregating feature scores from multiple patches. 
A two-stage pooling approach, including 3-norm pooling, is used to rank and aggregate the most important features and patches. 
The resulting vector of aggregated survival-discriminative features represents a WSI for a patient, contributing to the integrative deep learning model.
The genome and demography-specific layers in this model are adapted from the Cox-PASNet\cite{hao2018cox}, which is a pathway-based sparse deep neural network. 
The genome-specific layers consist of a gene layer, a pathway layer, and two hidden layers (H1 and H2). 
The gene layer serves as an input layer for gene expression data, with each node representing a gene. 
The pathway layer incorporates prior biological knowledge from databases like KEGG for biological interpretation. 
Connections between the gene layer and pathway layer are established based on biological pathway databases, with pathway nodes representing specific biological pathways. 
The two hidden layers capture nonlinear and hierarchical relationships between the pathways.
Clinical patient data are integrated into the demography-specific layer and combined with genomic features from gene expressions and aggregated survival-discriminative features from pathological images in the final hidden layer of the integrative model. 
To address overfitting in deep learning models with high-dimensional, low-sample-size data, the training technique from Cox-PASNet is applied. 
Instead of training the entire network, small networks are randomly selected, and sparse coding is used to create sparse connections for model interpretability. 
Training continues until convergence, with validation data used to monitor errors and prevent overfitting through early stopping.
PAGE-Net statistically outperformed Cox-EN with histopathological images only and Cox-PASNet with genomic data only. 

% Pathomic Fusion
In the model proposed in \cite{chen2020pathomic}, the cell graphs from histology images are supposed to get the cell-to-cell interactions and cell neighborhood structure.
To build cell graphs, the first step is generating accurate nuclei segmentation, in which a conditional generative adversarial network (cGAN) is used to learn the appropriate loss function for semantic segmentation.
The edge set and adjacency matrix of the graph are constructed using the K-Nearest Neighbors (KNN) algorithm from segmented cell nuclei.
In addition to manually computed statistics, an unsupervised technique called Contrastive Predictive Coding (CPC) is employed to extract 1024-dimensional features from tissue regions centered around each cell.
Graph Convolution Networks (GCNs) learn abstract feature representations for each node by aggregating feature vectors from their neighborhood through message passing.
In scenarios with a high-dimensional feature space and limited training samples, traditional feed-forward neural networks are susceptible to over-fitting. 
To address the challenge and apply more robust regularization techniques when training feed-forward networks on high-dimensional, low-sample-size genomics data, the model adopts normalization layers inspired by Self-Normalizing Networks introduced by Klambaeur et al\cite{klambauer2017self}.
Moreover, the Kronecker Product is used to construct a multi-modal representation. 
The feature vectors of histology images, cell graphs, and genomic features undergo matrix outer product operations to create a multi-modal tensor. 
This tensor captures important interactions among these three modalities in terms of single-modal, bimodal, and tri-modal relationships.
Ultimately, a neural network is trained using fully connected layers with the multi-modal tensor as input. 
The central aim of this method is to fuse heterogeneous modalities with distinct structural dependencies, thereby enhancing research and analysis in cancer pathology. 
To mitigate the impact of noisy uni-modal features during multi-modal training, a gating-based attention mechanism\cite{arevalo2017gated} is introduced to control the expressive power of features within each modality. 
When fusing histology images, cell graphs, and genomic features, the gating mechanism helps reduce the feature space's size before performing the Kronecker Product calculation.

% AMMASurv
To integrate the multi-modality data with different weights, the model proposed in \cite{wang2021ammasurv} uses an asymmetrical Transformer encoder.
The main idea to fuse other modality data unevenly is to add the new nodes and edges into the original graphs.
Different from the normal self-attention in Transformer, the noisy genomic nodes cannot impact the image features because they do not have the outgoing edges, which can only improve themselves by the influence of the imaging features.
% GPDBN
The GPDBN framework in \cite{wang2021gpdbn} comprises an inter-modality bilinear feature encoding module (Inter-BFEM) and two intra-modality bilinear feature encoding modules (Intra-BFEMs) to efficiently handle information interactions both across and within genomic data and pathological images.
This work needs further more data to enhance prognostic performance.

% MultiSurv
Similar to the MultimodalPrognosis mentioned, the model in \cite{vale2021long} can also process the missing data which includes tabular clinical data (herein simply referred to as “clinical”), gene expression (referred to as “mRNA”), microRNA expression (miRNA), DNA methylation (DNAm), gene copy number variation (CNV) data, and whole-slide images (WSI).
If some patients loss the whole modality data, they will be excluded when training the model using uni-modal data.
And the missing data will be replaced by zero matrix when training the multi-modal model.
To avoid over-fitting, the model has the mechanism to select some patients randomly and replace their specific modality data with zero input.
The t-SNE visualization reveals that patients with different cancer types occupy distinct clusters in a two-dimensional space, aligning well with known cancer type prognosis.
The model is trained end-to-end and delivers non-proportional outputs, achieving accurate long-term predictions across various cancer entities. 
Clinical data is identified as the most informative unimodal data modality.

% MCAT
Inspired by methods in Visual Question Answering (VQA), the model in \cite{chen2021multimodal} introduces new approach to let histology patches attend to genes in the survival prediction.
Unlike late fusion-based architectures that simply concatenate WSI-level bag representations with genomic features, the genomic-guided co-attention(GCA) layer captures multi-modal interactions, connecting histology-based visual concepts with gene embeddings, similar to the approach used in VQA.  
These interactions are visualized as attention heatmaps at the WSI level for each genomic embedding.
Additionally, the GCA layer reduces the effective "sequence length" of WSI bags from M instance-level patch features to N gene-guided visual concepts, where N represents the effective sequence length of gene embeddings (with M > N). 
This reduction enables more advanced feature aggregation techniques using self-attention and Transformers, allowing for supervision with entire WSIs, which was previously unattainable.
One limitation of the research was that they utilized a gene set that had been curated previously, and it contained genes with overlapping biological functional impact.

% PG-TFNet
The PG-TFNet proposed in \cite{lv2021pg} comprises three modules: a transformer-based multi-scale pathological feature fusion module, a cross-attention transformer-based multi-modal feature fusion module, and a final Cox layer for clinical data integration. 
The multi-scale feature fusion module processes pathological images at different magnification levels using transformer encoders, capturing relationships between image patches. 
The feature vectors from various magnification levels are concatenated, and a learnable class token and fixed positional encoding vector are introduced to create a multi-scale feature sequence. 
The multi-scale feature sequence is then input into a stack of transformer encoders for multi-scale feature fusion. 
This approach enables the extraction of morphological features at various field-of-view scales, improving the model's ability to understand the relationships between image patches.
The model utilizes a cross-attention transformer module to integrate pathological and genomic data. 
This module combines feature representations from both data types, enabling effective multi-modal data fusion for improved analysis of cancer prognosis.

% HFBSurv
To overcome the limitation of Kronecker product, HFBSurv extended GPDBN\cite{wang2021gpdbn} mentioned with the factorized bilinear model.
The model introduces a hierarchical multi-modal fusion approach, which employs factorized bilinear models to progressively integrate information from different levels, reducing computational complexity.  
Genomic data processing involves data cleaning, normalization, and feature selection using randomForestSRC\cite{yu2019breast}.
To process the missing values, the model uses the weighted nearest neighbors algorithm like \cite{ding2016evaluating}.
Pathological images are quantitatively analyzed to extract relevant features. 

% PORPOISE
The main contribution Richard et al. in \cite{chen2022pan} is the development of a research tool, the Pathology-Omics Research Platform for Integrative Survival Estimation (PORPOISE). 
PORPOISE is an interactive platform that provides prognostic markers learned by model for thousands of patients across 14 cancer types. 
It allows users to visualize H\&E images with interpretability overlays, local explanations of molecular features, and global patterns of feature importance. 
They used PORPOISE to analyze high-attention morphological regions in whole slide images and confirmed that the presence of tumor-infiltrating lymphocytes correlates with favorable cancer prognosis, as identified by their model. 
This tool facilitates the discovery of joint image-omic biomarkers.
PORPOISE consists of three network components: an attention-based multiple instance learning network (AMIL) for WSI inputs, a self-normalizing network (SNN) for molecular features, and a multimodal fusion layer (MMF) to integrate feature representations from AMIL and SNN.  
AMIL localizes prognostically relevant regions in WSIs without selective patch sampling and aggregates them for feature representation.  
SNN transforms molecular data into low-dimensional features.  
The model constructs a joint multimodal feature representation for histology and molecular data interactions, which is used for survival analysis.
One drawback of the platform is that while PORPOISE can elucidate "what," it may not always provide an explanation for "why."

% GC-SPLeM
In the model introduced in \cite{xie2022survival}, Xie et al. utilized a multimodal attention module to generate well-structured aggregated feature embeddings for patients and their association matrix. 
Using these embeddings as the foundation, they constructed a graph neural network for predicting survival outcomes.
Thanks to the superior graph quality and effective node aggregation process, the GNN model achieves precise predictions in both datasets with complete and incomplete data, offering a viable solution for handling the problem of missing data.
Each patient is treated as a node in a KNN-affinity graph, enabling the learning and encoding of topological structure and neighborhood information into each patient's feature vector.  
Unlike previous studies that treated patches of WSIs as nodes in a graph, this work innovatively constructs a graph of patients in datasets, motivated by the idea that patients with similar WSI and gene expression data are likely to have similar survival outcomes.  
This approach helps enhance patient representations, especially given the noisy and missing data often present in WSIs and gene expression datasets.

% ponet;pathway-based approaches
Apart from this, another approach to add the genotypic information is using the biological pathway databases to teach the model about the hidden biological functionality.
PONET proposed in \cite{qiu2023deep} uses a sparse biological pathway-informed embedding network for gene expression, additionally adopting the Multi-modal Factorized Bilinear pooling (MFB) method instead of original bilinear model to generate unimodal fusion to catch the modality-specific representations.
Getting each uni-modal fusion, the model can use the output representations as the input of the bimodal and tri-modal fusion respectively utilizing the bimodal attention and tri-modal attention.
Finally the model is trained through the Cox partial likelihood loss proposed by \cite{cheerla2019deep} used for the multi-modalities survival prediction to get the prediction results.

% HGCN
The model in \cite{hou2023hybrid} introduces multi-modal graphs as a foundation for a novel Hypergraph Convolution Network (HGCN) designed to extract prognostic-related information. 
The HGCN employs a node message passing mechanism for intra-modal interactions and a hyperedge mixing module for advanced modal interaction. 
Survival predictions were made by combining modalities using an online Multi-view Autoencoder (MAE) proposed in \cite{he2022masked} paradigm during model inference.
The research systematically explored the robustness of the multimodal survival prediction model, addressing an overlooked clinical challenge. 
To handle missing modalities, an online MAE method captures intrinsic dependencies and generates hyperedges. 
Several techniques are employed for handling missing modalities, including zero padding, multimodal factorized method (MFM) for reconstruction, autoencoders \cite{tran2017missing} \cite{liu2021incomplete}, and a lower-bound approach without the online MAE.
The experiments showed that the online MAE module played a key role in improving the prediction robustness.
The study underscored the potential of masked signal modeling for self-supervised learning across different scientific domains.

% DeepCoxSC



\subsection{WSI Features Fusion}

\subsection{GNN}
The extremely high gigapixel resolution of Whole Slide Images (WSI) requires researchers to divide them into smaller patches for analysis. Most of the previous work has primarily centered around aggregating images based on patches. But based on\cite{levy2020topological}, these methods may also overlook the crucial context between image patches and their neighbors for accurate predictions. Graphs are mathematical models depicting connections between pairs of elements. They are particularly effective for illustrating relationships between individual patches within a Whole Slide Image based on their spatial proximity or correlation. Currently, in risk prediction based on GNN through WSI images, many researchers have also achieved state-of-the-art results.
%DeepGraphSurv
DeepGraphSurv\cite{li2018graph}is the first GCN-based survival prediction model that uses WSIs as input. The authors believe that intermediate patch-wise features are a suitable choice for constructing a graph. They integrate global topology features and local patch features of WSIs using spectral convolution operators, which is the core of the entire architecture. In their approach, each patch is treated as a node, and node features are generated using a pre-trained VGG-16 model on ImageNet. The initial WSI graph is built based on patch features, and the VGG-16 feature extractor is not fine-tuned for WSI patches due to the absence of patch labels. Consequently, the initial graph may not accurately represent the topology between WSI patches, which is attributed to the limited training of feature networks.To address this issue, the authors design an independent graph G and  L to describe the topological relationships between specific survival-related WSI patches. This framework can simultaneously learn both local and global representations of Whole Slide Images by integrating local patch features with global topological structures through convolution. Typically, only a few local Regions of Interest (RoIs) in WSIs are relevant to survival analysis. Random sampling may not guarantee that all patches originate from RoIs. The attention mechanism is employed to selectively choose patches by learning their importance. So they introduce a parallel network with attention to adaptively learn attention on node features for selecting more important patches.
%Pact-CNN
To enhance the modeling of feature interaction between adjacent image patches during message transmission, Chen \cite{} introduced a context aware, spatially resolved patch based graph convolutional network. This network hierarchically aggregates instance level histological features to capture local and global topological structures in the tumor microenvironment
Their method uses Graph Convolutional Network (GCN), which iteratively aggregates and combines node features of different hidden layers through message passing. The message passing function of the network is adapted from DeepGCN, including message construction, permutation invariant aggregation, and update functions. Unlike previous graph based methods, they create neighborhoods based on the nearest neighbors in the embedded space, and construct graphs in Euclidean space. This allows them to use spatial convolution to perform local neighborhood aggregation functions similar to convolutional neural networks (CNNs). Compared with other methods of connecting nodes through adjacent image patches, Patch-GCN has improved performance, allowing for learning coarse-grained to fine-grained topological structures in tumor microenvironment.This model is suitable for any weakly supervised learning task in computational pathology that uses slide level or patient level labels, which helps to gain a more comprehensive understanding of representation learning in the tumor microenvironment.


\section{Conclusion}
A conclusion section is not required. Although a conclusion may review the 
main points of the paper, do not replicate the abstract as the conclusion.
A conclusion might elaborate on the importance of the work or suggest 
applications and extensions.

\appendices

\section*{Appendix and the Use of Supplemental Files}
Appendices, if needed, appear before the acknowledgment. If an appendix is not
critical to the main message of the manuscript and is included only for thoroughness
or for reader reference, then consider submitting appendices as supplemental materials.
Supplementary files are available to readers through IEEE \emph{Xplore\textregistered}
at no additional cost to the authors but they do not appear in print versions.
Supplementary files must be uploaded in ScholarOne as supporting documents, but for
accepted papers they should be uploaded as Multimedia documents. Refer readers
to the supplementary files where appropriate within the manuscript text using footnotes.
\footnote{Supplementary materials are available in the supporting documents/multimedia tab.
Further instructions on footnote usage are in the Footnotes section on the next page.}

\section*{Acknowledgment}
The preferred spelling of the word ``acknowledgment'' in American English is 
without an ``e'' after the ``g.'' Use the singular heading even if you have 
many acknowledgments. Avoid expressions such as ``One of us (S.B.A.) would 
like to thank $\ldots$ .'' Instead, write ``F. A. Author thanks $\ldots$ .'' In most 
cases, sponsor and financial support acknowledgments are placed in the 
unnumbered footnote on the first page, not here.


\bibliography{ref}
\bibliographystyle{unsrt}
\end{document}

